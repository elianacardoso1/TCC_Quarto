% Capa
\thispagestyle{empty}
\pagenumbering{roman}

\begin{center}
    \textbf{\MakeUppercase{$institution$}}\\[0.5em]
    \textbf{$setor$}\\
    \textbf{$Departamento$}
\end{center}


\vspace{2.0cm}


\begin{center}
$author$

\vspace{6.cm}

{\textbf{\MakeUppercase{$title$}}}\\[1em]
{\normalsize $subtitle$}
\end{center}

\vspace{9.0cm}






\begin{center}

$city$ 

$year$
    
\end{center}

\newpage




% Folha de aprovação
\thispagestyle{empty}

\begin{center}
$author-upper$
\end{center}

\vspace{3.0cm}

\begin{center}
    {\textbf{\MakeUppercase{$title$}}}\\[1em]
    {\normalsize $subtitle$}
\end{center}

\vspace{3.0cm}

\begin{adjustwidth}{7.5cm}{0cm}
    \justifying
    $confirm_text$
\end{adjustwidth}

\vspace{1.0cm}

%APROVADA: $confirm_date$

\vspace{3.5cm}

\centering
\rule{6cm}{0.4pt} \\
$advisor-name$ \\
Orientadora



\vspace{0.5cm}

\centering
\rule{6cm}{0.4pt} \\
$advisor-name2$ \\
Co-orientadora

\vspace{2.0cm}

\begin{center}

$city$ 

$year$
    
\end{center}
\newpage


%\pagestyle{fancy}
% Dedicatoria

\begin{invisible}
.
\end{invisible}

\vspace{28\baselineskip}

\begin{adjustwidth}{7.5cm}{0cm}
    \justifying
    \singlespacing
    $dedication$
\end{adjustwidth}


\newpage
\pagestyle{fancy}

\newcommand{\resumo}{%
\titleformat{\section}[block]{\bfseries\filcenter}{}{1em}{}
}
\resumo

\section*{AGRADECIMENTO}
\justify

\begin{invisible}
.
\end{invisible}

Agradeço primeiramente a Deus, pela saúde, força e fé que me sustentaram durante toda esta caminhada.

Aos meus pais, Nadil e Luiza, por todo amor, apoio incondicional e por sempre me apoiarem. Às minhas irmãs, por serem minhas melhores amigas.

Aos meus amigos e colegas, pelas conversas, risadas e por me lembrarem de que eu não estava sozinha nesse percurso. Àqueles que torceram por mim, mesmo de longe, o meu carinho e gratidão.

À minha orientadora, Professora Ilka Afonso Reis, pela paciência em me ensinar. Sua orientação foi essencial para a realização deste trabalho.

Agradeço também aos professores e funcionários do Departamento de Estatística da UFMG, pela formação sólida, pelo ambiente de aprendizado e acolhimento —  e à FUMP, que me deu coragem para me mudar para BH com o suporte financeiro que tornou tudo possível.

Por fim, agradeço a mim mesma — por não desistir. Cada desafio superado neste percurso contribuiu não apenas para esta monografia, mas para a pessoa que me tornei.

Muito obrigada!

\newpage


% Resumo
\pagestyle{fancy}

\section*{RESUMO}


\begin{spacing}{1.0}
\justify

\setlength{\parindent}{0pt}


Esta monografia tem como objetivo avaliar os impactos da escolha metodológica entre a regressão linear mista e a regressão logística ordinal com efeitos aleatórios na modelagem de variáveis ordinais obtidas por escalas de Likert. Para isso, foi reaplicado o modelo de regressão linear utilizado por Prates et al. (2022) e realizada uma reanálise com a regressão logística ordinal mista, utilizando o mesmo banco de dados. O estudo original investigava o efeito de diferentes modelos de Rotulagem Nutricional Frontal (RNF) — octógono, triângulo e lupa — na percepção de saudabilidade de alimentos supostamente saudáveis por consumidores brasileiros.

Os resultados mostraram que a regressão logística ordinal mista é mais adequada para variáveis com estrutura ordinal, especialmente em contextos com hierarquia nos dados e quando se trata de escalas de Likert. Esse modelo respeita a natureza dos dados e permite a interpretação direta por meio das odds ratios, que não dependem do tamanho da escala. Além disso, foi capaz de capturar efeitos principais e interações que o modelo linear não identificou.

Em termos substantivos, observou-se que todos os modelos de RNF aumentaram a percepção de que os produtos eram pouco saudáveis, sobretudo quando apresentavam nutrientes críticos em excesso e ausência de alegações. Entre os rótulos, o octógono e o triângulo se destacaram como os mais eficazes, em comparação ao modelo da lupa — atualmente adotado no Brasil.

Conclui-se que o modelo ordinal misto oferece uma alternativa metodológica mais robusta para dados provenientes de escalas de classificação, ao mesmo tempo em que os resultados do estudo reforçam a eficácia das estratégias de rotulagem frontal como instrumento de apoio à promoção da saúde pública.

\textbf{Palavras-chave:} $keywords$

\setlength{\parindent}{15pt}

\end{spacing}

\newpage
% Lista de figuras
\renewcommand{\listfigurename}{LISTA DE FIGURAS}
\pagestyle{fancy}
\listoffigures

\newpage
% Lista de tabelas
\renewcommand{\listtablename}{LISTA DE TABELAS}
\pagestyle{fancy}
\listoftables

\newpage
% Sumário
\renewcommand{\contentsname}{SUMÁRIO}
\pagestyle{fancy}
\tableofcontents

\titleformat{\section}{\normalsize\bfseries}{\thesection.}{1em}{}
\newpage
\pagenumbering{arabic}